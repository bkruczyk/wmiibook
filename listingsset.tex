%% konfifuracja wyglądu listingów
%% źródło: http://stackoverflow.com/questions/741985/latex-source-code-listing-like-in-professional-books
%% więcej informacji: ftp://ftp.tex.ac.uk/tex-archive/macros/latex/contrib/listings/listings.pdf

\usepackage{color}
\usepackage{listings}

%% wygląd treści listingów
\lstset{
    basicstyle=\footnotesize,           %\ttfamily, % domyślna czcionka
    %% numbers=left,                    % strona po której zostaną wstawione numery wierszy
    numberstyle=\tiny,                  % styl numeracji wiersze
    %% stepnumber=1,                    % krok numeracji wierszy
    numbersep=5pt,                      % odstęp numeracji wierszy od tekstu
    tabsize=2,                          % rozmiar tabulacji
    extendedchars=true,                 %
    breaklines=true,                    % czy łamać wiersze
    keywordstyle=\color{black},
    frame=b,
    keywordstyle=[1]\textbf,                    % styl słów kluczowych
    %% keywordstyle=[2]\textbf,                 %
    %% keywordstyle=[3]\textbf,                 %
    %% keywordstyle=[4]\textbf,                 %
    %% stringstyle=\color{white}\ttfamily       % styl stringów
    stringstyle=\ttfamily,
    showspaces=false,                           % czy pokazywać spacje
    showtabs=false,                             % czt pokazywać tabulatory
    xleftmargin=17pt,
    framexleftmargin=17pt,
    framexrightmargin=5pt,
    framexbottommargin=4pt,
    %% backgroundcolor=\color{lightgray},
    showstringspaces=false                      % czy pokazywać spacje w stringach
}

%% wygląd nagłowków listingów
\usepackage{caption}
\DeclareCaptionFont{white}{\color{white}}
\DeclareCaptionFormat{listing}{\colorbox[cmyk]{0.43, 0.35, 0.35,0.01}{\parbox{\textwidth}{\hspace{15pt}#1#2#3}}}
\captionsetup[lstlisting]{format=listing,labelfont=white,textfont=white,singlelinecheck=false,margin=0pt,font={bf,sf,footnotesize}}

%% polska nazwa nagłówka
\renewcommand{\lstlistlistingname}{Spis listingów}
